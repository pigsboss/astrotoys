\documentclass{article}
\usepackage{amsmath}
\usepackage{mathrsfs}
\usepackage{hyperref}
\begin{document}
\title{Notes on orbital mechanics}

\maketitle
\section{Keplerian orbit and orbital elements}
\textbf{Keplerian orbit} is the motion of one body (the orbiting body)
relative to another (the central body) if there is only
point-like gravitational interaction between the two bodies.

\textbf{Orbital elements} are parameters that uniquely identify a specific orbit. 
There are six orbital elements, where two for the geometry of the ellipse
(or quadratic curve such as parapola and hyperbola in a more general context),
three for the orientation of the orbit in a 3D coordinate system and one
for the position of the celestial body.

\textbf{Eccentricity} ($e$) and \textbf{semimajor axis} ($a$)
define the shape and size of the orbit.
\textbf{Longitude of ascending node} ($\Omega$),
\textbf{inclination} ($i$) and
\textbf{argument of periapsis} ($\omega$)
are considered as the proper Euler angles in $z-x'-z''$ order,
which rotate the orbit from
\begin{equation}
  \frac{(x+c)^2}{a^2} + \frac{y^2}{b^2} = 1\text{,}
  \label{equ:ellipse}
\end{equation}
lying on the $xOy$ to an arbitrary plane.
Either \textbf{True anomaly} ($\nu$), \textbf{mean anomaly} ($M$) or
\textbf{eccentric anomaly} ($E$) indicates the immediate position of
the celestial body.


\section{Specific orbital energy}
The \textbf{specific orbital energy} $\epsilon$ of two bodies is the contant
sum of their mutual potential energy, which is zero at infinite distance, and
their total kinetic energy, divided by the reduced mass.
\begin{equation}
  \epsilon = \frac{v^2}{2} - \frac{\mu}{r}\text{,}
\end{equation}
where $v$ is relative orbital speed, $\mu$ is the standard gravitational parameter
and $r$ is the orbital distance between the two bodies.

The \textbf{characteristic energy} $C_3$ is a measure of the excess specific orbital energy over
that required to just barely escape from the central body.
\begin{equation}
  C_3 = v^2 - \frac{2 \mu}{r} = 2 \epsilon\text{.}
\end{equation}

The semi-major axis $a$ can be calculated from $\epsilon$ as
\begin{equation}
  a = -\frac{\mu}{2 \epsilon}\text{.}
\end{equation}

When $\epsilon < 0$ the orbiting body is captured by the central body and orbiting it on
an \textbf{elliptical orbit}, where the semi-major axis $a > 0$.
When $\epsilon = 0$ the orbiting body has exactly the energy needed to escape from the
central body and is leaving it on a \textbf{parabolic trajectory},
where $a \to \infty$.
When $\epsilon > 0$ the orbiting body has more than enough energy to escape the central
body and is leaving it on a \textbf{hyperbolic trajectory}, where $a < 0$.

Elliptical (including circular) orbit is called \textbf{capture orbit}
while parabolic and hyperbolic trajectories are called \textbf{escape orbits}.

\section{Anomalies}
\subsection{True anomaly}
\textbf{True anomaly} $\nu$ is the angle between the direction of periapsis and the current
position of the celestial body, as seen from the main focus of the orbit.

\subsection{Mean anomaly}
For an elliptical orbit, \textbf{mean anomaly} $M$ is the angle between the direction of the
periapsis and the current position of a fictitious body which moves in a circular
orbit with constant speed in the same orbital period as the actual celestial body.

However, for an escape orbit, the motion is not periodic, so mean anomaly is generally
defined as
\begin{equation}
  M = \sqrt{\frac{\mu}{\vert a \vert ^3}}(t - \tau)\text{,}
  \label{equ:mean-anomaly}
\end{equation}
where $\tau$ is the \textbf{epoch}.

Semi-major axis $a \to \infty$ for parabolic trajectory, so mean anomaly here is defined as
\begin{equation}
  M_{\mathrm{parabola}} = \sqrt{\frac{\mu}{\ell^3}}(t - \tau)\text{,}
  \label{equ:mean-anomaly-parabola}
\end{equation}
where $\ell$ is semi-latus rectum.

\subsection{Eccentric anomaly}
For any point on the elliptical orbit specified by Equation \ref{equ:ellipse},
textbf{eccentric anomaly} $E$ is introduced as
\begin{equation}
  \begin{cases}
    x = a\cos E - c \\
    y = b\sin E
  \end{cases}\text{.}
\end{equation}
Given the distance
\begin{equation}
  \sqrt{x^2+y^2} = r = \frac{a(1 - e^2)}{1 + e \sin \nu}\text{,}
\end{equation}
the eccentric anomaly can be calculated from the true anomaly as
\begin{equation}
  \cos E = \frac{e + \cos \nu}{1 + e \cos\nu}\text{.}
  \label{equ:eccentric-anomaly}
\end{equation}

\subsection{Hyperbolic anomaly}
Hyperbolic anomaly $H$ is analogous to eccentric anamoly $E$
(as in Equation \ref{equ:eccentric-anomaly}) as
\begin{equation}
  \cosh H = \frac{e + \cos \nu}{1 + e\cos \nu}\text{,}
\end{equation}
as well as
\begin{equation}
  \begin{cases}
    x = a \cosh H - c \\
    y = b \sinh H
  \end{cases}\text{.}
\end{equation}


\section{Kepler's equation}
Kepler's equation is used to calculate the position of the orbiting
body in a Keplerian orbit.

First calculate mean anomaly $M$ at given time with Equation \ref{equ:mean-anomaly}.
Then calculate eccentric anomaly $E$ from mean anomaly $M$ by solving Kepler's equation
for elliptical orbit
\begin{equation}
  M = E - e \sin E\text{,}
\end{equation}
with Newton's method, i.e.,
\begin{equation}
  E_{n+1} = E_n - \frac{E_n - e \sin E_n - M}{1 - e \cos E_n}\text{,}
\end{equation}
where $E_0 = M$.
For hyperbolic trajectory calculate hyperbolic anomaly $H$ by solving hyperbolic
Kepler's equation
\begin{equation}
  M = e \sinh H - H\text{.}
\end{equation}
When using Newton's method in solving hyperbolic Kepler's equation there are
situations where it requires division by $0$ so the following fixed point iteration
is preferred:
\begin{equation}
  H_{n+1} = \mathrm{arcsinh}\frac{H_{n} + M}{e}\text{.}
\end{equation}
Once either eccentric anomaly or hyperbolic anomaly is calculated the cartesian
coordinate of the orbiting body can be directly calculated.

\section{Barker's equation}
Solve Barker's equation which relates the time of flight to true anomaly of a parabolic trajectory
\begin{equation}
  6 \sqrt{\frac{\mu}{\ell^3}}(t - \tau) = 3 \tan\frac{\nu}{2} + \tan^3\frac{\nu}{2}\text{,}
  \label{equ:barker}
\end{equation}
where $\ell$ is semi-latus rectum, $\nu$ is true anomaly.

With the substitutions
\begin{align}
  M &= \frac{1}{2}\sqrt{\frac{\mu}{\ell^3}}(t - \tau)\quad\text{and} \\
  N &= \sqrt[3]{3 M + \sqrt{9 M^2 + 1}}\text{,}
\end{align}
Equation \ref{equ:barker} can be solved directly
\begin{equation}
  \nu = 2 \arctan\left(N - \frac{1}{N}\right)\text{.}
\end{equation}
\end{document}
